\section{Approximate Jacobian Models}
\label{sec:jacobian_models}

Two use two different Jacobian approximation methods in this these; a diminishing rigidity Jacobian and an adaptive Jacobian, which are described below.

\subsection{Diminishing Rigidity Jacobian}
\label{sec:diminishing_rigidity}

The key assumption used by this method~\cite{Berenson2013} is \textit{diminishing rigidity}: the closer a gripper is to a particular part of the deformable object, the more that part of the object moves in the same way that the gripper does (i.e. more ``rigidly''). The further away a given point on the object is, the less rigidly it behaves; the less it moves when the gripper moves. This approximation depends on two parameters $\drktrans \geq 0$ and $\drkrot \geq 0$ which control how the translational and rotational rigidity scales with distance. Small values entail very rigid objects; high values entail very deformable objects.

For every point $\deformidx$ and every gripper $\gripperidx$ we construct a Jacobian $\Jrigid(\deformidx, \gripperidx)$ such that if $\deformconfigI$ was rigidly attached to the gripper $\gripperconfigG$ then
\begin{equation}
    \deformvelI = \JrigidIG \grippervelG = 
    \begin{bmatrix} \JtransIG & \JrotIG \end{bmatrix} \grippervelG \enspace .
\end{equation}
We then modify this Jacobian to account for the effects of \textit{diminishing rigidity}. Let the set of points grasped by gripper $\gripperidx$ be $\GraspedPointsFull \subseteq \deformconfig$. Then for 

Let $\closestpointIG$ be the index of the point with minimial relaxed geodesic distance to $\deformconfigI$ among the ones grasped by gripper $\gripperidx$:
\begin{equation}
    \closestpointIG = \argmin_{j \in }
\end{equation}

Let $\geodistIG$ be a measure of the distance between gripper $\gripperidx$ and point $\deformidx$. Then the translational rigidity of point $\deformidx$ with respect to gripper $\gripperidx$ is defined as
\begin{equation}
    \rigiditytransIG = e^{-\drktrans \geodistIG}
\end{equation}
and the rotational rigidity is defined as
\begin{equation}
    \rigidityrotIG = e^{-\drkrot \geodistIG}.
\end{equation}
To construct an approximate Jacobian $\tilde \Jacobian(\deformidx, \gripperidx)$ for a single point and a single gripper we combine the rigid Jacobians with their respective rigidity values
\begin{equation}
    \tilde \Jacobian(\deformidx, \gripperidx) = \begin{bmatrix} \rigiditytransIG \JtransIG & \rigidityrotIG \JrotIG \end{bmatrix} \enspace,
\end{equation}
and then combine the results into a single matrix
\begin{equation}
    \tilde \Jacobian(\gripperconfig, \deformconfig) = 
    \begin{bmatrix}
        \tilde \Jacobian(1,1) & \tilde \Jacobian(1,2) & \dots & \tilde \Jacobian(1, \ngrippers) \\
        \tilde \Jacobian(2,1) & \ddots \\
        \vdots \\
        \tilde \Jacobian(\ndeformpoints,1)
    \end{bmatrix} \enspace .
\end{equation}


\subsection{Adaptive Jacobian}
\label{sec:adaptive_jacobian}

A different approach is taken in~\cite{NavarroAlarcon2014}, instead using online estimation to approximate $\Jacobian(\gripperconfig, \deformconfig)$.
In this formulation we start with some estimate of the Jacobian $\tilde \Jacobian(0)$ at time $t = 0$ and then use the Broyden update rule~\cite{Broyden1965} to update $\tilde \Jacobian(t)$ at each timestep $t$
\begin{equation}
    \tilde \Jacobian(t) = \tilde J(t-1) + \ajrate \frac{\left( \deformvel(t) - \tilde \Jacobian(t-1) \grippervel(t) \right)}{\grippervel(t)^T \grippervel(t)} \grippervel(t)^T \enspace.
\end{equation}
This update rule depends on a update rate $\ajrate \in (0, 1]$ which controls how quickly the estimate shifts between timesteps.