\section{Modelling Deformable Objects}

Deformable object manipulation has been studied in many contexts ranging across domestic, surgical, and industrial domains \cite{Jimenez2012}. Much of this work relies on accurate modeling and simulation of deformable objects. Some of the most common simulation methods use Mass-Spring Models \cite{Essahbi2012, Maris2010} and Finite-Element Models (FEM) \cite{Muller2002}, However, such approaches usually require significant tuning and are very sensitive to discretization parameters. Also, we seek a model that can be evaluated very quickly inside an optimal control framework, and Finite-element models, while accurate, can be computationally-expensive to simulate. While methods have been developed to track objects using FEM in real-time \cite{Petit2017}, a controller may need to evaluate the model many times to find an appropriate command, requiring speeds faster than real-time. Specialized models have also been developed, e.g., \cite{Borum2014} and \cite{Bretl2014} focus on elastic rods that are not in contact. We seek a model that works well with rope- and cloth-like materials that can deform as a result of contact. Finally, researchers have also investigated automatic modeling of deformable objects \cite{Lang2002, Cretu2008}. However, these methods rely on a time-consuming training phase for each object to be modeled, which we would like to avoid.%and it requires a lot of efforts in preparing test cases.

Given a model such as those above, researchers have investigated various control methods to manipulate deformable objects. %Some controllers are designed with an explicit model to represent the deformable object. 
Hirai and Wada proposes an iterative visual-servoing controller in \cite{Hirai2000}, which aligns interest points on the deformable object to targets. 
Its control law is based on modeling the deformable object as a lattice of interconnected springs. 
\cite{Wada2001} uses a similar spring model to formulate a PID controller that aligns interest points to targets. While effective, these methods rely on models like those above and thus suffer from the above issues.
%All these controllers rely on an explicit model to represent the deformable object which can induce the high computation cost and heavy parameters tuning work.


Our work is complementary to methods that adapt the model of the object during manipulation \cite{Navarro-Alarcon2014, NavarroAlarcon2018, Hu2018deformable_gpr}. Our model can serve as an initial guess and a reference for such methods so that the online adaptation process does not diverge too far from a reasonable model as a result of perception or modeling error. 
%However these models require a good initial guess of the Jacobian to begin the adaptation process. The work proposed here could serve as such a guess. %Instead of an exact model of the deformable object, they explore the features of the dynamics from the task space. 
%The philosophy of the deformable object modeling in this paper is very close to these works. 
Our paper builds on the idea of diminishing rigidity Jacobians \cite{Berenson2013} by improving the model and formulating appropriate constraints for deformable object manipulation. \cite{McConachie2018} extended \cite{Berenson2013} to use multiple Jacobian-based models when it is uncertain which model is appropriate for the object. The new model proposed here could be incorporated into the framework of \cite{McConachie2018}. 
%and study the methods to tune the Jacobian-based model, as well as the way to maximize the rewards of tasks by selecting the right controller for operation.


% manipulation model in the B.M
The model in Sec.~\ref{sec:constrained_model} is built on the method for deformable object manipulation in \cite{Berenson2013} (Sec.~\ref{sec:diminishing_rigidity}) and task space control of robot manipulators \cite{Khatib1987}.
However, unlike \cite{Berenson2013}, we explore the directional diminishing rigidity of the deformable object. This maintains a low computational cost of prediction while improving accuracy.
%Our approach to handling obstacle avoidance is similar to \cite{Berenson2013} and \cite{Sentis2005} in that it repels the gripper from obstacles using the surface normal. As a result, the prediction of object movements will not interfere with the obstacle avoidance constraint.
