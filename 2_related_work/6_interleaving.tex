\section{Task and Motion Planning}

Robotic manipulation of deformable objects has been studied in many contexts ranging from surgery to industrial manipulation (see \cite{Khalil2010} and \cite{Sanchez2018deformablesurvey} for extensive surveys). Below we discuss the most relevant methods to the work presented here, starting with methods of simulating and planning for deformable objects. We then discuss visual servoing and learning-based methods for similar tasks. In addition to previous work in deformable object manipulation, we also discuss related work in planning/control for robot arms and ways to consider topology in planning, which we draw from for our framework. We end with a discussion of probabilistic completeness and describe why previous methods to show this property do not apply, motivating our proof method.

Much work in deformable object manipulation relies on simulating an accurate model of the object being manipulated. Motivated by applications in computer graphics and surgical training, many methods have been developed for simulating string-like objects \citep{Bergou2008,Rungjiratananon2011} and cloth-like objects \citep{Baraff1998,Goldenthal2007}. The most common simulation methods use Mass-Spring models \citep{Gibson1997, Essahbi2012}, which are generally not accurate for large deformations \citep{Maris2010}, and Finite-Element (FEM) models \citep{Muller2002,Irving2004,Kaufmann2008}. FEM-based methods are widely used and physically well-founded, but they can be unstable when subject to contact constraints, which are especially important in this work. They also require significant tuning and are very sensitive to the discretization of the object. Furthermore, such models require knowledge of the physical properties of the object, such as it's Young's modulus and friction parameters, which we do not assume are known.

Motion planning for manipulation of deformable objects is an active area of research \citep{Jimenez2012}. \citet{Saha2008} present a Probabilistic Roadmap (PRM) \citep{Kavraki1996} that plans for knot-tying tasks with rope. \citet{Rodriguez2006} study motion planning in fully deformable simulation environments. Their method, based on Rapidly-exploring Random Trees (RRTs) \citep{LaValle2006}, applies forces directly to an object to move it through narrow spaces while using the simulator to compute the resulting deformations. \citet{Frank2011} presented a method that pre-computes deformation simulations in a given environment to enable fast multi-query planning. Other sampling-based approaches have also been proposed \citep{Anshelevich2000a,BurchanBayazit2002,Gayle2005,Lamiraux2001,Moll2006,Roussel2015}. However, all the above methods either disallow contact with the environment or rely on potentially time-consuming physical simulation of the deformable object, which is often very sensitive to physical and computational parameters that may be difficult to determine. In contrast our method uses simplified models for control and motion planning with far lower computational cost. In addition, the use of a local controller is not considered in the above methods, instead relying on a global planner (and thus implicitly the accuracy of the simulator) to generate a path that completes the entire task.

Model-based visual servoing approaches bypass planning entirely, and instead use a local controller to determine how to move the robot end-effector for a given task \citep{Hirai2000,Smolen2009,Wada2001}. Our recent work \citep{Berenson2013,McConachie2018} as well as \cite{Navarro-Alarcon2014,NavarroAlarcon2016,NavarroAlarcon2018} bypass the need for an explicit deformable object model, instead using approximations of the Jacobian to drive the deformable object to the attractor of the starting state. More recent work by \citet{Hu2018deformable_gpr} has enabled the use of Gaussian process regression while controlling a deformable object. Rather than using only a planner or only a controller, our framework uses both components, each when appropriate.


Approaches based on learning from demonstration avoid planning and deformable object modelling challenges entirely by using offline demonstrations to teach the robot specific manipulation tasks \citep{Huang2015,Schulman2016}; however, when a new task is attempted a new training set needs to be generated. In our application we are interested in a way to manipulate a deformable object without a high-fidelity model or training set available \textit{a priori}. For instance, imagine a robot encountering a new piece of clothing for a new task. While it may have models for previously-seen clothes or training sets for previous tasks, there is no guarantee that those models or training sets are appropriate for the new task.


\citet{Park2014Interleaving} considered interleaving planning and control for arm reaching tasks in rigid unknown environments. In their method, they assume an initially unknown environment in which they plan a path to a specific end-effector position. This path is then followed by a local controller until the task is complete, or the local controller gets stuck. If the local controller gets stuck, then a new path is planned and the cycle repeats. In contrast, our controller is performing the task directly rather than following a planned reference trajectory, incorporating deadlock prediction into the execution loop, while our global planner is planning for both the robot motion as well as the deformable object stretching constraint.

Our planning method has some similarity to topological \citep{Bhattacharya2012,Jaillet2008} and tethered robot \citep{Brass2015,SoonkyumKim2015} planning techniques; these methods use the topological structure of the space to define homotopy classes, either as a direct planning goal, or as a way to help inform planning in the case of tethered robots. Planning for some deformable objects, in particular rope or string, can be viewed as an extension of the tethered robot case where the base of the tether can move. This extension, however, requires a very different approach to homotopy than is commonly used, particularly when working in three-dimensional space instead of a planar environment. In our work we use \textit{visiblity deformations} from \cite{Jaillet2008} as a way to encode homotopy-like classes of configurations.

Previous approaches to proving probabilistic completeness for efficient planning of underactuated systems rely on the existence of a steering function to move the system from one region of the state space to another, or choosing controls at random \citep{LaValle2001,Karaman2013,Kunz2015,LiAOKP2016}. For deformable objects, a computationally-efficient steering function is not available, and using random controls can lead to prohibitively long planning times. \citet{Roussel2015} bypass this challenge by analyzing completeness in the submanifold of quasi-static contact-free configurations of a extensible elastic rods. In contrast, we show that our method is probabilistically complete even when contact between the deformable object and obstacles is considered along the path. Note that it is especially important to allow contact at the goal configuration of the object to achieve coverage tasks. \citet{LiAOKP2016} present an efficient asymptotically-optimal planner which does not need a steering function, however, they do rely on the existence of a contact free trajectory where every point in the trajectory is in the interior of the valid configuration space. Our proof of probabilistic completeness is based on \citet{LiAOKP2016}, but we allow for the deformable object to be in contact with obstacles along a given trajectory.

