\chapter{Interleaving Planning and Control}
\label{chap:interleaving}

The previous chapters have focused on local methods for solving tasks. While we've shown that these methods are capable of performing interesting tasks, they are unable to escape from local minima due to their very design. This chapter is focused on a method to overcome this limitation by adding planning to the set of tools that we can apply to a given task.

One of the challenges in planning for deformable object manipulation is the high number of degrees of freedom involved; even approximating the configuration of a piece of cloth in 3D with a 4 $\times$ 4 grid results in a 48 degree of freedom configuration space. In addition, the dynamics of the deformable object are difficult to model \citep{Essahbi2012}; even with high-fidelity modeling and simulation, planning for an individual task can take hours \citep{Bai2016}. Local controllers on the other hand are able to very efficiently generate motion, however, they are only able to successfully complete a task when the initial configuration is in the ``attraction basin'' of the goal as seen in Chapter~\ref{chap:local_control}.

\begin{figure*}[t]
    \centering
    \includegraphics[trim={5cm 6cm 3cm 4cm},clip,width=0.24\textwidth,height=1.5in,keepaspectratio]{single_pole_starting_configuration}\hfill
    \includegraphics[trim={2cm 3cm 6cm 7cm},clip,width=0.24\textwidth,height=1.5in,keepaspectratio]{double_slit_starting_configuration}\hfill
    \includegraphics[trim={0cm 1cm 0cm 1cm},clip,width=0.24\textwidth,height=1.5in,keepaspectratio]{rope_maze_starting_configuration}\hfill
    \includegraphics[trim={1.5cm 0cm 0.6cm 0cm},clip,width=0.24\textwidth,height=1.5in,keepaspectratio]{live_robot_starting_configuration}%
    \caption{Four example manipulation tasks for our framework. In the first two tasks, the objective is to cover the surface of the table (indicated by the red lines) with the cloth (shown in green). In the first task, the grippers (shown in blue) can freely move however the cloth is obstructed by a pillar. In the second task, the grippers must pass through a narrow passage before the table can be covered. In the third task, the robot must navigate a rope (shown in green in the top left corner) through a three-dimensional maze before covering the red points in the top right corner. The maze consists of top and bottom layers (purple and green, respectively). The rope starts in the bottom layer and must move to the target on the top layer through an opening (bottom left or bottom right). For the fourth task, the physical robot must move the cloth from the far side of an obstacle to the region marked in pink near the base of the robot.}
    \label{fig:example_tasks}
\end{figure*}

The central question we address in this work is how can we combine the strengths of global planning with the strengths of local control while mitigating the weakness of each? We propose a framework for interleaving planning and control which uses global planning to generate gross motion of the deformable object, and a local controller to refine the configuration of the deformable object within the local neighborhood. By separating planning from control we are able to use different representations of the deformable object, each suited to efficient computation for their respective roles. In order to determine when to use each component, we introduce a novel deadlock prediction algorithm that is inspired by topologically-based motion planning methods \citep{Bhattacharya2012,Jaillet2008}. By answering the question ``Will the local controller get stuck?'' we can predict if the local controller will be unable to achieve the task from the current configuration. If we predict that the controller will get stuck we can then invoke the global planner, moving the deformable object into a new neighbourhood from which the local controller may be able to succeed. The key to our efficient prediction is forward-propagating only the stretching constraint, assuming the object will otherwise comply to contact.

We seek to solve problems for one-dimensional and two-dimensional deformable objects (i.e. rope and cloth) where we need to arrange the object in a particular way (e.g. covering a table with a tablecloth) but where there is also complex environment geometry preventing us from directly completing the task. While we cannot claim to solve all problems in this class (in particular in environments where the deformable object can be snagged), we can still solve practical problems where the path of the deformable object is obstructed by obstacles. In this work we restrict our focus to controllers of the form described in Section~\ref{sec:local_control}, and tasks suited to these controllers. Examples of these types of tasks are shown in Figure~\ref{fig:example_tasks}. In our experiments we show that this iterative method of interleaving planning and control is able to successfully perform several interesting tasks where our planner or controller alone are unable to succeed.

Our contributions are: (1) A novel deadlock prediction algorithm to determine when a global planner is needed; (2) An efficient and probabistically-complete global planner for rope and cloth manipulation tasks; and (3) A framework to combine local control and global motion planning to leverage the strengths of each while mitigating their weaknesses. We present experiments in both a simulated environment and on a physical robot (Figure~\ref{fig:example_tasks}). Our results suggest that our planner can efficiently find paths, taking under a second on average to generate a feasible path in three out of four simulated scenarios. The physical experiment shows that our framework is able to effectively perform tasks in the real world, where reachability and dual-arm constraints make the planning more difficult.

%%%%%%%%%%%%%%%%%%%%%%%%%%%%%%%%%%%%%%%%%%%%%%%%%%%%%%%%%%%%%%%%%%%%%%%%%%%%%%%%%%%%%%%%%%%%%%%%%%%%%%%%%%%%%%%%%%%%%%%

\section{Problem Statement}
\label{sec:main_problem_statement}

\todoin{This duplicates some previous stuff. Refer back or keep here?}
Define the robot configuration space to be $\robotCspace$. We assume that the robot configuration can be measured exactly. Denote an individual robot configuration as $\robotconfig \in \robotCspace$. This set can be partitioned into a valid and invalid set. The valid set is referred to as $\robotCvalid$, and is the set of configurations where the robot is not in collision with the static geometry of the world. The invalid set is referred to as $\robotCinvalid = \robotCspace \setminus \robotCvalid$.

\todoin{Confirm the `previous chapters' statement.}
We assume that our model of the robot is purely kinematic, with no higher order dynamics. Previous chapters assumed an arbitrary number of grippers; in this chapter we restrict the problem to cases where two end-effectors that are rigidly attached to the object. We assume that the robot moves slowly enough that we can treat the combined robot and deformable object as quasi-static. Let the function $\DeformForwardFn(\robotconfig, \deformconfig, \robotvel)$ map the system configuration $(\robotconfig, \deformconfig)$ and robot movement $\robotvel$ to the corresponding deformable object movement $\deformvel$.

Similar to the previous chapter, we define a task based on a set of $\ntargetpoints$ target points $\target \subset \reals^3$, a function $\ErrorFnFull \rightarrow \reals^{\geq 0}$, which measures the alignment error between $\deformconfig$ and $\target$, and a termination function $\TermCondFull$ which indicates if the task is finished. Let a robot controller be a function $\CtrlFull$\footnote{A specific controller may have additional parameters (such as gains in a PID controller), but we do not include such parameters here to keep $\Ctrl(\dots)$ in a more general form.} which maps the system state $\left( \robotconfig, \deformconfig \right)$ and alignment targets $\target$ to a desired robot motion $\robotvelcmd$. In this work we restrict our discussion to tasks and controllers of the form introduced in Chapter~\ref{chap:local_control}; these controllers are local, i.e. at each time $t$ they choose an incremental movement $\robotvelcmd$ which reduces the alignment error as much as possible at time $t + 1$. 

The problem we address in this chapter is how to find a sequence of $\taskexecutiontime$ robot commands $\robotvelcmdexpanded = \robotvelcmdseq$ such that each motion is feasible, i.e. it should not bring the grippers into collision with obstacles, should not cause the object to stretch excessively, and should not exceed the robot's maximum velocity $\robotvelmax$. Let these feasibility constraints be represented by $A(\robotvelcmdseq) = 0$. Then the problem we seek to solve is:
\begin{equation}
    \begin{aligned}
        & \text{find}   & & \taskexecutiontime, \robotvelcmdseq \\
        & \text{s.t.}   & & \TermCond(\target, \deformconfig_{\taskexecutiontime}) = \texttt{true}\\ 
        &               & & A(\robotvelcmdseq) = 0\\
    \end{aligned}
    \label{eqn:veb_general_problem}
\end{equation}
where $\deformconfig_{\taskexecutiontime}$ is the configuration of the deformable object after executing $\robotvelcmdseq$.

Solving this problem directly is impractical in the general case for two major reasons. First, modeling a deformable object accurately is very difficult in the general case, especially if it contacts other objects or itself. Second, even given a perfect model, computing precise motion of the deformable object requires physical simulation, which can be very time consuming inside a planner/controller where many potential movements need to be evaluated. We seek a method which does not rely on high-fidelity modelling and simulation; instead we present a framework combining both global planning and local control to leverage the strengths of each in order to efficiently perform the task.

%%%%%%%%%%%%%%%%%%%%%%%%%%%%%%%%%%%%%%%%%%%%%%%%%%%%%%%%%%%%%%%%%%%%%%%%%%%%%%%%%%%%%%%%%%%%%%%%%%%%%%%%%%%%%%%%%%%%%%%

\section{Interleaving Planning and Control}
\label{sec:main_framework_loop}

Global planners are effective at finding paths through complex configuration spaces, but for highly underactuated systems such as deformable objects achieving a specific configuration is very difficult even with high-fidelity models; this means that we cannot rely on them to complete a task independent of a local controller. In order for the local controller to complete the task, the system must be in the correct basin of attraction. From this point of view it is not the planner's responsibility to complete a task but rather to move the system into the right basin for the local controller to finish the task. By explicitly separating planning from control we can use different representations of the deformable object for each component; this allows us to use a highly-simplified model of the deformable object for global planning to generate gross motion of the deformable object, while using an independent local approximation for the controller. The key question then is when should we use global planning versus local control?

Our framework can be broken down into three major components: (1) A global motion planner to generate gross motion of the deformable object; (2) A local controller for refinement of the configuration of the deformable object; and (3) A novel deadlock prediction algorithm to determine when to use planning versus control. Figure~\ref{fig:main_loop_diagram} shows how these components are connected, switching between a local controller loop and planned path execution loop as needed. In the following sections we describe each component in turn, starting with the local controller.

\begin{figure}[h]
    \centering
    \includegraphics[width=0.8\columnwidth]{FrameworkBlockDiagram}
    \caption{Block diagram showing the major components of our framework. On each cycle we use either the local controller (dotted purple arrows) or a planned path (dashed red arrows) to predict if the system will be deadlocked in the future, planning a new path is needed to avoid deadlock.}
    \label{fig:main_loop_diagram}
\end{figure}


\subsection{Local Control}
\label{sec:local_control}

The role of the local controller is not to perform the whole task, but rather to refine the configuration of the deformable object locally. For our local controller we use a controller of the form introduced in Section~\ref{sec:stretching_avoidance_controller}. These controllers locally minimize error $\ErrorFn$ while avoiding robot collision and excessive stretching of the deformable object.

An important limitation of this approach is that the individual navigation functions used by these controllers are defined and applied independently of each other; this means that the navigation functions that are combined to define the direction to move the deformable object can cause the controller to move the end effectors on opposite sides of an obstacle, leading to poor local minima, i.e. becoming stuck. Figure~\ref{fig:overstretch_example} shows our motivating example of this type of situation. Other examples of this kind of situation are shown in Section~\ref{sec:simulation_experiments}. In addition, while this local controller prevents collision between the robot and obstacles, it does not explicitly have any ability to \textit{go around} obstacles.

\begin{figure}[h]
    \centering
    \includegraphics[width=\columnwidth]{OverstretchExample}
    \caption{Motivating example for deadlock prediction. The local controller moves the grippers on opposite sides of an obstacle, while the geodesic between the grippers (red line) cannot move past the pole, eventually leading to overstretch or tearing of the deformable object if the robot does not stop moving towards the goal.}
    \label{fig:overstretch_example}
\end{figure}

In order to address these limitations we introduce a novel deadlock prediction algorithm to detect when the system $(\robotconfig_t, \deformconfig_t)$ is in a state that will lead to deadlock (i.e. becoming stuck) if we continue to use the local controller.

\subsection{Predicting Deadlock}
\label{sec:predicting_deadlock}

Predicting deadlock is important for two reasons; first we do not want to waste time executing motions that will not achieve the task. Second, we want to avoid the computational expense of planning our way out of a cul-de-sac after reaching a stuck state. By predicting deadlock before it happens we address both of these concerns. The key idea is to detect situations similar to Figure~\ref{fig:overstretch_example} where the local controller will wrap the deformable object around an obstacle without completing the task. We also need to detect situations where no progress can be made due to an obstacle directly in the path of the desired motion of the robot.

Let $\truemotionfull = \robotvelact$ be the true motion of the robot when $\robotvelcmd$ is executed for unit time; in this section we will be predicting the future state of the system, thus it is not sufficient to consider only $\robotvelcmd$, we must also consider $\robotvelact$.  Modelling inaccuracies as well as the deformable object being in contact can lead to meaningful differences between $\robotvelcmd$ and $\robotvelact$. Specifically, when a deformable object is in contact with the environment, tracking $\robotvelcmd$ perfectly may lead to a constraint violation (i.e. overstretch or tearing of the deformable object).

We consider a controller to be deadlocked if the commanded motion produces (nearly) no actual motion, and the task termination condition is not met:
\begin{equation}
    \begin{split}
        \| \robotvelact_t \|                &\approx 0 \\
        \TermCond(\target, \deformconfig_t) &= \texttt{false}.
        \label{eqn:stuck}
    \end{split}
\end{equation}
In general we cannot predict if the system will get stuck in the limit; to do so would require a very accurate simulation of the deformable object. Instead we predict if the system will get stuck within a prediction horizon $\predictionhorizon$ timesteps. We divide our deadlock prediction algorithm into three parts and discuss each in turn: 1) estimating gross motion; 2) predicting overstretch; and 3) progress detection.

\subsubsection{Estimating Gross Motion}

The idea central to our prediction (Alg.~\ref{alg:predict_deadlock}) is that while we may not be able to determine precisely how a given controller will steer the system, we can capture the gross motion of the system and estimate if the controller will be deadlocked. We split the prediction into two parts; first we assume that controller $\Ctrl$ is able to manipulate the deformable object with a reasonable degree of accuracy within a local neighborhood of the current state. This allows us to approximate the motion of the deformable object by following the task-defined navigation functions for each $\deformconfigI \in \deformconfig$. Examples of this approximation are shown in Figure~\ref{fig:gross_deformable_motion}.

\begin{algorithm}[h]
\caption{PredictDeadlock$(\ErrorFn, \robotconfig_t, \deformconfig_t, \band_t, \target, \predictionhorizon, \textrm{Path})$}
\begin{algorithmic}[1]
    \State ConfigHistory $\gets [\textrm{ConfigHistory}, q_t]$
    \State ErrorHistory $\gets [\textrm{ErrorHistory}, \ErrorFn(\deformconfig_t)]$
    \State BandPredictions $\gets []$
    \State $\correspondences \gets$ CalculateCorrespondences$(\deformconfig_t, \target)$
    \For{$n = t, \dots, t + \predictionhorizon - 1$}
        \If {Path $\neq \emptyset$}
            \State $\deformvel_e, \Pinvweight_e \gets$ FollowNavigationFunction$(\deformconfig_n, \correspondences)$
            \State $\deformconfig_{n + 1} \gets \deformconfig_n + \deformvel_e$
            \State $\robotvelcmd_n \gets$ FindBestRobotMotion$(\robotconfig_n, \deformconfig_n, \deformvel_e, \Pinvweight_e)$
            \State $\robotconfig_{n + 1} \gets \robotconfig_n + \robotvelcmd_n$
        \Else
            \State $\robotconfig_{n + 1} \gets \robotconfig_n + $ FollowPath(Path)
        \EndIf
        \State $\band_{n + 1} \gets$ ForwardPropagateBand$(\band_n, \robotconfig_{n + 1})$
        \State BandPredictions $\gets [\textrm{BandPredictions}, \band_{n + 1}]$
    \EndFor
    \If {PredictOverstretch$(\textrm{BandPredictions})$ \textbf{or} \\ NoProgress$($ConfigHistory, ErrorHistory$)$}
        \State \Return \texttt{true}
    \Else
        \State \Return \texttt{false}
    \EndIf
\end{algorithmic}
\label{alg:predict_deadlock}
\end{algorithm}


Next we use a simplified version of LocalController() which omits the stretching avoidance terms (Alg.~\ref{alg:stretching_avoidance_controller} lines 3 and 4) to predict the commands sent to the robot. These terms are omitted as they can be sensitive to the exact configuration of the deformable object, which is not considered in this approximation. If we are executing a path then we can use the planned path directly to predict overstretch.

\subsubsection{Predicting Overstretch}
\label{sec:overstretch}

Next we introduce the notion of a \textit{virtual elastic band} (VEB) between the robot's end-effectors. This VEB represents the shortest path through the deformable object between the end-effectors. The band approximates the constraint imposed by the deformable object on the motion of the robot; if the end-effectors move too far apart, then the VEB will be too long, and thus the deformable object is stretched beyond a task-specified maximum stretching factor $\stretchmax$. Similarly, if the VEB gets caught on an obstacle and becomes too long, then the deformable object is also overstretched. By considering only the geodesic between the end-effectors, we are assuming that the rest of deformable object will comply to the environment, and does not need to be considered when predicting overstretch. The VEB representation allows us to use a fast prediction method, but does not account for the part of the material that is slack. We discuss this trade-off further in Chapter \ref{chap:learning_when_to_trust}. This VEB is based on Quinlan's path deformation algorithm~\cite{Quinlan1994} and is used both in deadlock prediction as well as global planning (Section~\ref{sec:planning_goal} and Section~\ref{sec:global_planning})


Denote the configuration of an VEB at time $t$ as a sequence of $\nbandpoints$ points $\band_t \subset \reals^3$. The number of points used to represent an VEB can change over time, but for any given environment and deformable object there is an upper limit $\nbandpointsmax$ on the number of points used. Define $\textrm{Path}(\band)$ to be the straight line interpolation of all points in $\band$. Define the length of a band to be the length of this straight line interpolation. At each timestep the VEB is initialized with the shortest path between the end effectors through the deformable object, and then ``pulled'' tight using the internal contraction force described in ~\cite{Quinlan1994} Section~5, and a hard constraint for collision avoidance. The endpoints of the band track the predicted translation of the end effectors (Alg.~\ref{alg:band_propogation}). This band represents the constraint that must be satisfied for the object not to tear. By considering only this constraint on the object in prediction, we are implicitly relying on the object to comply to contact as it is moved by the robot. We discuss the limitations of this assumption in the discussion (Section~\ref{sec:planning_discussion}).

\begin{algorithm}[h]
\caption{ForwardPropagateBand$(\band, \robotconfig)$}
\begin{algorithmic}[1]
    \State $(p_0, p_1) \gets$ ForwardKinematics$(\robotconfig)$
    \State $\band \gets$ $[p_0, \band, p_1]$
    \State $\band \gets$ InterpolateBandPoints$(\band)$
    \State $\band \gets$ RemoveExtraBandPoints$(\band)$
    \State $\band \gets$ PullTight$(\band)$
    \State \Return $\band$
\end{algorithmic}
\label{alg:band_propogation}
\end{algorithm}

Let $\bandlength_{t+n}$ be the length of the path defined by the VEB $\band_{t+n}$ at timestep $n$ in the future, and $\bandlengthmax$ be the longest allowable band length. To use this length sequence to predict if the controller will overstretch the deformable object, we perform three filtering steps: an annealing low-pass filter, a filter to eliminate cases where the band is in freespace, and the detector itself which predicts overstretch. We use a low-pass annealing filter with annealing constant $\bandlengthannealing \in [0, 1)$ to mitigate the effect of numerical and approximation errors which could otherwise lead to unnecessary planning:
\begin{equation}
    \begin{split}
        \tilde \bandlength_{t + 1} &= \bandlength_{t + 1} \\
        \tilde \bandlength_{t + n} &= \bandlengthannealing \tilde \bandlength_{t + n - 1} + (1 - \bandlengthannealing) \bandlength_{t + n} \enspace ,  n = 2, \dots, \predictionhorizon \enspace .
    \end{split}
\end{equation}
Second, we discard from consideration any bands which are not in contact with an obstacle; we can eliminate these cases because our local controller includes an overstretch avoidance term which will prevent overstretch in this case in general. Last we compare the filtered length of any remaining band predictions to $\bandlengthmax$; if after filtering, there is an estimated band length $\tilde \bandlength$ that is larger than $\bandlengthmax$ then we predict that the local controller will be stuck. An example of this type of detection is shown in Figure~\ref{fig:overstretch_predicted}, where the local controller will wrap the cloth around the pole, eventually becoming deadlocked in the process.

\begin{figure}[h]
    \centering
    \includegraphics[width=0.5\columnwidth]{GrossDeformableMotion}
    \caption{Example of estimating the gross motion of the deformable object for a prediction horizon $\predictionhorizon = 10$. The magenta lines start from the points of the deformable object that are closest to the target points (according to the navigation function). These lines show the paths those points would follow to reach the target when following the navigation function.}
    \label{fig:gross_deformable_motion}
\end{figure}

\begin{figure}[h]
    \centering
    \includegraphics[width=0.5\columnwidth]{OverstretchPredicted}
    \caption{Estimated gross motion of the deformable object (magenta lines) and end effectors (blue spheres). The VEB (black lines) is forward propagated by tracking the end effector positions, changing to cyan lines when overstretch is predicted.}
    \label{fig:overstretch_predicted}
\end{figure}

\subsubsection{Progress Detection}

Last, we track the progress of the robot and task error to estimate if the controller $\Ctrl$ is making progress towards the task goal. This is designed to detect cases when the robot is trapped against an obstacle. Naively we could look for instances when $\robotvelact = 0$ however due to sensor noise, actuation error, and using discrete math in a computer, we need to use a threshold instead. At the same time we want to avoid false positives, where the robot is moving slowly but task error is decreasing. To address these concerns we record the configuration of the robot (stored in ConfigHistory) and the task error (stored in ErrorrHistory) every time we check for deadlock, and introduce three parameters to control what it means to be making progress: history window $\historywindow$, error improvement threshold $\errorprogressthreshold$, and configuration distance threshold $\motionprogressthreshold$. If over the last $\historywindow$ timesteps, the improvement in error is less than $\errorprogressthreshold$, and the robot has moved less than $\motionprogressthreshold$, then we predict that the controller will not be able to reach the goal from the current state and trigger global planning.


\subsection{Setting the Global Planning Goal}
\label{sec:planning_goal}


In order to enable efficient planning, we need to approximate the configuration of the deformable object in a way that captures the gross motion of the deformable object without being prohibitively expensive to use. We use the same approach from Section~\ref{sec:overstretch}, but the interpretation in this use is slightly different; the VEB is a proxy for the leading edge of the deformable object. In this way we can plan to move the deformable object to a different part of the workspace without needing to simulate the entire deformable object, instead the deformable object conforms to the environment naturally.

In order to make progress towards achieving the task, we want to set the goal for the global planner to be a configuration that we have not explored with the local controller. We do so in two parts; we find the set of all target points $\target_U$ which are contributing to task error, split these points into two clusters, and use the cluster centers to define the goal region of the end effectors, $\eepositiongoal$; any end-effector position within a task-specified distance $\eegoalradius$ is considered to have reached the end-effector goal (Alg.~\ref{alg:call_global_planner} lines 1-3). Second, we set the goal configuration of the VEB to be any configuration that is not similar to a \textit{blacklist} of VEBs. This blacklist is the set of all band configurations from which we predicted that the local controller would be deadlocked in the future (Section~\ref{sec:predicting_deadlock}).

\begin{algorithm}[h]
\caption{PlanPath$(\robotconfig_t, \deformconfig_t, \band_t, \target, \textrm{Blacklist})$}
\begin{algorithmic}[1]
    \State $\target_U \gets$ UncoveredTargetPoints$(\target, \deformconfig_t)$
    \State $\eepositiongoal \gets$ ClusterCenters$(\target_U)$
    \State $\eepositiongoal \gets$ ProjectOutOfCollision$(\eepositiongoal)$
    \State $\bandgoal \gets \{\band \mid \text{VisCheck}(\band, \text{Blacklist}) = 0\}$
    \State Path $\gets$ RRT-EB$(\robotconfig_t, \band_t, \eepositiongoal)$
    \If {Path $\neq$ Failure}
        \State \Return ShortcutSmooth$($Path, $\bandgoal)$
    \Else
        \State \Return Failure
    \EndIf
\end{algorithmic}
\label{alg:call_global_planner}
\end{algorithm}

To define similarity we use Jaillet and Sim\'{e}on's \textit{visibility deformation} definition to compare two VEBs (\cite{Jaillet2008}). Intuitively two VEBs are similar if you can sweep a straight line connecting the two bands from the start points to the end points of the two bands without intersecting an obstacle. Unlike the original use, we do not constrain the start and end points of each path to match, but the algorithm is identical. We use this as a heuristic to find states that are dissimilar from states where we have already predicted that the local controller would be deadlocked. Let VisCheck$(\band, \textrm{Blacklist}) \rightarrow \{0, 1\}$ denote this visibility deformation check, returning $1$ if $\band$ is similar to a band in the blacklist and $0$ otherwise. Then
\begin{equation}
    \bandgoal = \{\band \mid \text{VisCheck}(\band, \textrm{Blacklist}) = 0\}
    \label{eqn:bandgoal}
\end{equation}
is the set of all VEBs that are dissimilar to the Blacklist.

Combined, $\eepositiongoal, \eegoalradius$, and $\bandgoal$ define what it means for the planner to have found a path to the goal (Alg.~\ref{alg:goal_check}); the end-effectors must be in the right region, and the VEB must be dissimilar to any band in the Blacklist.

\begin{algorithm}[h]
\caption{GoalCheck$(\rrtnodeset, \eepositiongoal, \bandgoal)$}
\begin{algorithmic}[1]
    \For {$\reducedstate = (\robotconfig, \band) \in \rrtnodeset$}
        \State $\eeposition \gets$ ForwardKinematics$(\robotconfig)$
        \If {$\| \eeposition[0] - \eepositiongoal[0] \| \leq \eegoalradius$ \textbf{and} 
             $\| \eeposition[1] - \eepositiongoal[1] \| \leq \eegoalradius$ \textbf{and} 
             $\band \in \bandgoal$}
            \State \Return \texttt{true}
        \EndIf
    \EndFor
    \State \Return \texttt{false}
\end{algorithmic}
\label{alg:goal_check}
\end{algorithm}


The combination of local control, deadlock prediction, and global planning are shown in the MainLoop function (Alg.~\ref{alg:interleaving_mainloop}). Because the VEB is an approximation we need to predict deadlock while executing the planned path. We use the same prediction method for path execution as for the local controller. To set the maximum band length $\bandlengthmax$ used by the global planner and the deadlock prediction algorithms, we calculate the geodesic distance between the grippers through the deformable object in its ``laid-flat'' state and scale it by the task specified maximum stretching factor $\stretchmax$.

% \todoin{Check consistency of parameter inputs to functions in alg blocks}

\begin{algorithm}[h]
\caption{MainLoop$(\target, \TermCond, \ErrorFn, \deformconfig_\textrm{relaxed}, \stretchmax)$}
\begin{algorithmic}[1]
    \State $\RelaxedDistMatrix \gets$ GeodesicDistanceBetweenEndEffectors$(\deformconfig_\textrm{relaxed})$
    \State $\bandlengthmax \gets \stretchmax \RelaxedDistMatrix$
    \State Blacklist $\gets \emptyset$
    \State Path $\gets \emptyset$
    \State $t \gets 0$
    \State $\robotconfig_0 \gets$ SenseRobotConfig$()$
    \State $\deformconfig_0 \gets$ SensePoints$()$
    
    \While{$\neg \TermCond(\target, \deformconfig_t)$}
        \State $\band_t \gets$ InitializeBand$(\deformconfig_t)$
        
        \If {PredictDeadlock$(\ErrorFn, \robotconfig_t, \deformconfig_t, \band_t, \target, \textrm{Path})$}
            \State Blacklist $\gets$ Blacklist $\cup \{ \band_t \}$
            \State Path $\gets$ PlanPath$(\robotconfig_t, \deformconfig_t, \band_t, \target, \textrm{Blacklist})$
            \If {Path = Failure}
                \State \Return Failure
            \EndIf
        \EndIf
        
        \If {Path $\neq \emptyset$}
            \State $\robotvelcmd \gets$ FollowPath(Path)
            \If {PathFinished(Path)}
                \State Path $\gets \emptyset$
            \EndIf
        \Else
            \State $\robotvelcmd \gets$ LocalController$(\robotconfig_t, \deformconfig_t, \target, \RelaxedDistMatrix, \stretchmax)$
        \EndIf
        
        \State CommandConfiguration$(\robotconfig_t + \robotvelcmd)$
        \State $\robotconfig_{t+1} \gets$ SenseRobotConfig$()$
        \State $\deformconfig_{t+1} \gets$ SensePoints$()$
        \State $t \gets t + 1$
    \EndWhile
    \State \Return Success
\end{algorithmic}
\label{alg:interleaving_mainloop}
\end{algorithm}


%%%%%%%%%%%%%%%%%%%%%%%%%%%%%%%%%%%%%%%%%%%%%%%%%%%%%%%%%%%%%%%%%%%%%%%%%%%%%%%%%%%%%%%%%%%%%%%%%%%%%%%%%%%%%%%%%%%%%%%

\section{Global Planning}
\label{sec:global_planning}

The purpose of the global planner is not to find a path to a configuration where the task is complete, but rather to move the system into a state from which the local controller can complete the task. Planning directly in configuration space of the full system $\cspace_r \times \deformconfigspace$ is not practical for two important reasons. First, this space is very high-dimensional and the system is highly underactuated. More importantly, to accurately know the state of the deformable object after a series of robot motions one would need a high-fidelity simulation that has been tuned to represent a particular task. We seek to plan paths very quickly without knowing the physical properties of a deformable object \textit{a priori}. The key idea that allows us to plan paths quickly is to consider only the constraint on robot motion that is imposed by the deformable object; i.e. the robot motion shall not tear or cause excessive stretching of the deformable object. We represent this constraint using a virtual elastic band and enforce the constraint that the band's length cannot exceed $\maxbandlength$.


\subsection{Planning Setup}

% This work examines both the configuration space of just the robot $\cspace_r$ and the configuration space of the robot and band $\cspace_f = \cspace_r \times \bandspace$.  Both of these sets can be partitioned into a valid and invalid set.  For the robot configuration space, the valid set is referred to as \rev{$\cfree_r$}, and is the set of configurations where the robot is not in collision with the static geometry of the world.  \rev{T}he invalid set is referred to as $\cinv_r = \cspace_r \setminus \cfree_r$.

Denote the planning configuration space as $\cspace_f = \cspace_r \times \bandspace$. In order to split $\cspace_f$ into valid and invalid sets, we first define what it means for a band $\band \in \bandspace$ to be valid. A band $\band \in \bandspace$ is considered valid if the band is not overstretched and the path defined by $\band$ does not \textit{penetrate} an obstacle:
\begin{equation}
    \bandspacevalid = \{ \band \mid \textrm{Length}(\band) \leq \maxbandlength \textrm{ and } 
                                    \textrm{Path(\band)} \cap \textrm{Interior}(\obstacle) = \emptyset \} \enspace .
\end{equation}
Then the invalid set is $\bandspaceinv = \bandspace \setminus \bandspacevalid$. Similarly define $\cfree_f = \cfree_r \times \bandspacevalid$ and $\cinv_f = \cspace_f \setminus \cfree_f$.




\begin{algorithm}[t]
\caption{MainLoop$(\deformtarget, \terminationcondition, \errorfunction, \deformconfig_\textrm{flat}, \maxstretchfactor, \stretchingcorrectionweightfactor,\predictionhorizon, \goalreachradius,\bestneardist)$}
\begin{algorithmic}[1]
    \State $\relaxeddistancematrix \gets$ GeodesicDistanceBetweenEndEffectors$(\deformconfig_\textrm{flat})$
    \State $\maxbandlength \gets \maxstretchfactor \relaxeddistancematrix$
    \State Blacklist $\gets \emptyset$
    \State Path $\gets \emptyset$
    \State $t \gets 0$
    \State $\robotconfig[0] \gets$ SenseRobotConfig$()$
    \State $\deformconfig_0 \gets$ SensePoints$()$
    
    \While{$\neg \terminationcondition(\deformconfig_t)$}
        \State $\band_t \gets$ InitializeBand$(\deformconfig_t)$
        
        \If {PredictDeadlock$(\errorfunction, \robotconfig[t], \deformconfig_t, \band_t, \deformtarget, \maxbandlength, \predictionhorizon,\textrm{Path})$}
            \State Blacklist $\gets$ Blacklist $\cup \{ \band_t \}$
            \State Path $\gets$ PlanPath$(\robotconfig[t], \deformconfig_t, \band_t, \deformtarget, \goalreachradius, \maxbandlength, \bestneardist, \textrm{Blacklist})$
            \If {Path = Failure}
                \State \Return Failure
            \EndIf
        \EndIf
        
        \If {Path $\neq \emptyset$}
            \State $\robotcommandvel \gets$ FollowPath(Path)
            \If {PathFinished(Path)}
                \State Path $\gets \emptyset$
            \EndIf
        \Else
            \State $\robotcommandvel \gets$ LocalController$(\robotconfig[t], \deformconfig_t, \deformtarget, \relaxeddistancematrix, \maxstretchfactor, \stretchingcorrectionweightfactor)$
        \EndIf
        
        \State CommandConfiguration$(\robotconfig[t] + \robotcommandvel)$
        \State $\robotconfig[t+1] \gets$ SenseRobotConfig$()$
        \State $\deformconfig_{t+1} \gets$ SensePoints$()$
        \State $t \gets t + 1$
    \EndWhile
    \State \Return Success
\end{algorithmic}
\label{alg:interleaving_mainloop}
\end{algorithm}






$\cspace_r$ and $\cspace_f$ are imbued with distance metrics $d_r(\cdot,\cdot) : \cspace_r \times \cspace_r \rightarrow \reals^{\geq 0}$ and $d_f(\cdot,\cdot) : (\cspace_r \times \bandspace) \times (\cspace_r \times \bandspace) \rightarrow \reals^{\geq 0}$, respectively. We define distances in robot configuration space and band space to be additive. I.e.
\begin{equation}
    d_f(\cdot, \cdot)^2 = d_r(\cdot, \cdot)^2 + \banddistscale d_b(\cdot, \cdot)^2
    \label{eqn:dist_metric}
\end{equation}
for some scaling factor $\banddistscale > 0$. To measure distances in $\bandspace$, we first upsample each band using linear interpolation to use the maximum number of points $\maxbandpoints$ for the given task, then measure the Euclidean distance between the upsampled points when considered as a single vector (Alg.~\ref{alg:band_dist}).

For a given planning problem, we are given a query $( \qinit_f, \Qgoal_f )$ which describes the initial configuration of the robot and band, as well as a goal region for the system to reach.  Note that $\Qgoal_f$ is defined implicitly via the GoalCheck() function and the parameters $(\eepositiongoal, \goalreachradius,\textrm{ Blacklist})$ rather than any explicit enumeration.


\begin{algorithm}[t]
\caption{BandDistance: $d_b(\band_1, \band_2)$}
\begin{algorithmic}[1]
    \State $\tilde \band_1 \gets$ UpsamplePoints$(\band_1, \maxbandpoints)$
    \State $\tilde \band_2 \gets$ UpsamplePoints$(\band_2, \maxbandpoints)$
    \State \Return $\| \tilde \band_1 - \tilde \band_2\|$
\end{algorithmic}
\label{alg:band_dist}
\end{algorithm}

%Before describing the properties of $\rpath_f$, a few additional definitions and assumptions must be provided. 

We now establish a relationship between a path in robot configuration space $\cspacepath_r$ and one in the full configuration space $\cspacepath_f$ by making the following assumption.


\begin{assumption}[Deterministic Propagation]
\label{ass:deterministic}
    Given an initial configuration in full space $\qinit_f \in \cfree_f$ and the corresponding robot configuration $\qinit_r \in \cfree_r$, a path $\cspacepath_r : [0, 1] \rightarrow \cfree_r$ in robot configuration space with $\cspacepath_r(0) = \qinit_r$ uniquely defines a single path in full space $\cspacepath_f$, where $\cspacepath_f(0) = \qinit_f$.  Specifically, define
    \begin{equation}
        \cspacepath_f(t) =\\ \begin{bmatrix} \cspacepath_r(t) \\ \lim_{h \rightarrow 0^-} \textup{ForwardPropogateBand}(\band(t-h), \cspacepath_r(t)) \end{bmatrix} \enspace .
        \label{eqn:deterministic}
    \end{equation}
\end{assumption}

Eq.~\eqref{eqn:deterministic} implicitly defines an underactuated system where the only way we can change the state of the band is by moving the robot; for a path in the full configuration space $\cspacepath_f$ to be achievable there must be a robot configuration space path $\cspacepath_r$, which when propagated using Eq.~\eqref{eqn:deterministic}, produces $\cspacepath_f$. Let $\textrm{FullSpace}(\cspacepath_r, \qinit_f)$ be the function that maps a given robot configuration space path $\cspacepath_r$ and full space initial configuration $\qinit_f$ to the full space path defined by Eq.~\eqref{eqn:deterministic}.




\subsection{Planning Problem Statement}

For a given planning instance, the task is to find a path starting from $\qinit_f$ through $\cfree_f$ to any point in $\Qgoal_f$, while obeying the constraints implied by Eq.~\eqref{eqn:deterministic}.

For a sequence of robot configurations $\qinit_f, \config_{1,r}, \dots, \config_{M,r} \in \cspace_r$, let 
\begin{equation}
    \cspacepath_r = \textrm{Path}(\qinit_r, \config_{1,r}, \dots, \config_{M,r})    
\end{equation}
be the path defined by linearly interpolating between each point in order. Then, formally, the problem our planner addresses is the following:
\begin{equation}
    \begin{aligned}
        & \text{find}   & & \{ \config_{1,r}, \dots, \config_{M,r} \} \\
        & \text{s.t.}   & & \cspacepath_r = \textrm{Path}(\qinit_r, \config_{1,r}, \dots, \config_{M,r}) \\
        &               & & \cspacepath_f(s) \in \cfree_f, \; \forall s \in [0, 1] \\
        &               & & \cspacepath_f(1) \in \Qgoal_f \enspace .
    \end{aligned}
    \label{eqn:planning_problemstatement}
\end{equation}
\noindent where $\cspacepath_f = \textrm{FullSpace}(\cspacepath_r, \qinit_f)$.


\subsection{RRT-EB}

\begin{algorithm}[t]
\caption{RRT-EB$(\robotconfig[t], \band_t, \eepositiongoal, \goalreachradius, \bandgoal, \maxbandlength, \bestneardist, \goalbias)$}
\begin{algorithmic}[1]
    \State $\rrtnodeset \gets \{(\robotconfig[t], \band_t)\}$
    \State $\rrtedgeset \gets \emptyset$
    \State $\Qapproxgoal \gets$ GetGoalConfigs$(\eepositiongoal)$
    
    \While {$\neg$MaxTimeEllapsed()}
            \State $\qrand_f \gets$ SampleUniformConfig() \label{alg:bandrrt:basic_start}
            \State $\qnear_f \gets$ BestNearest$(\rrtnodeset, \rrtedgeset, \bestneardist, \qrand_f)$
            \State $\nodesnew, \edgesnew \gets$ Connect$(\qnear_f, \qrand_r, \maxbandlength)$
            \State $\rrtnodeset \gets \rrtnodeset \cup \nodesnew$
            \State $\rrtedgeset \gets \rrtedgeset \cup \edgesnew$
            
            \If {GoalCheck$(\nodesnew, \eepositiongoal, \goalreachradius, \bandgoal) = 1$}
                \State \Return ExtractPath$(\rrtnodeset, \rrtedgeset)$
            \EndIf \label{alg:bandrrt:basic_end}
            
            \State $\gamma \sim$ Uniform$[0, 1]$ \label{alg:bandrrt:bias_start}
            \If {$\gamma \leq \goalbias$}
                \State $\qlast_f \gets$ LastConfig$(\qnear_f, \nodesnew)$ \label{alg:bandrrt:lastconfig}
                \State $\qbias \gets \argmin_{\robotconfig \in \Qapproxgoal} d_r(\qlast_r, \robotconfig)$
                \State $\nodesnew, \edgesnew \gets$ Connect$(\qlast_f, \qbias, \maxbandlength)$
            \State $\rrtnodeset \gets \rrtnodeset \cup \nodesnew$
            \State $\rrtedgeset \gets \rrtedgeset \cup \edgesnew$
                
                \If {GoalCheck$(\nodesnew, \eepositiongoal, \goalreachradius, \bandgoal) = 1$}
                    \State \Return ExtractPath$(\rrtnodeset, \rrtedgeset)$
                \EndIf
            \EndIf \label{alg:bandrrt:bias_end}
    \EndWhile
    \State Return Failure
\end{algorithmic}
\label{alg:bandrrt}
\end{algorithm}

\begin{algorithm}[t]
\caption{BestNearest$(\rrtnodeset, \rrtedgeset, \bestneardist, \qrand_f)$}
\begin{algorithmic}[1]
    \State $\Qnear_f \gets \{ \config_f | \config_f \in \rrtnodeset, d_f(\config, \qrand_f) \leq \bestneardist \}$
    \If {$\Qnear_f \neq \emptyset$}
        \State \Return $\argmin_{\rrtnode \in \rrtnodeset} \cost(\rrtnode, \rrtnodeset, \rrtedgeset)$
    \Else
        \State $\Dnear[r]^2 \gets \min_{\config \in \rrtnodeset}{d_r(\qrand, \robotconfig)^2}$ \label{alg:nearst:robotspace}
        \State $\Dmax[f]^2 \gets \Dnear[r]^2\ + \banddistscale \Dmax[b]^2$
        \State $\Qnear_f \gets \{ \config | \config \in \rrtnodeset, d_r(\robotconfig, \qrand_r)^2 \leq \Dmax[f]^2 \}$ \label{alg:nearest:radius}
        \State \Return $\argmin_{\config_f \in \Qnear_f}{d_f(\config_f, \qrand_f)}$ \label{alg:nearest:fullspace}
    \EndIf
\end{algorithmic}
\label{alg:nearest}
\end{algorithm}


Our planner, RRT for Elastic Bands (RRT-EB), (Alg.~\ref{alg:bandrrt}) is based on an RRT with changes to account for a virtual elastic band in addition to the robot configuration. Lines \ref{alg:bandrrt:basic_start}-\ref{alg:bandrrt:basic_end} perform random exploration with lines \ref{alg:bandrrt:bias_start}-\ref{alg:bandrrt:bias_end} biasing the tree expansion towards the goal region. The key variations are the BestNearest function (Alg.~\ref{alg:nearest}) and the goal bias method.

BestNearest is based on the selection method used by~\cite{LiAOKP2016}, selecting the node of smallest cost within a radius $\bestneardist$ if one exists, falling back to standard nearest neighbour behaviour if no node in the tree is within $\bestneardist$ of the random sample. We use path length in robot configuration space $\cspace_r$ as a cost function in our implementation. This helps reduce path length and ensures that we can specify lower bounds in Sec.~\ref{sec:delta_sim_traj_construction}. In order to avoid calculating distances in the full configuration space when it is not necessary, our method for finding the nearest neighbor is split into two parts, first searching in robot space, then searching in the full configuration space (see Fig.~\ref{fig:nearest}). Sec.~\ref{sec:nn_equiv} shows that this method is equivalent to searching in the full configuration space directly. $\bestneardist$ is an additional parameter compared to a standard RRT; it controls how much focus is placed on path cost versus exploration. The smaller $\bestneardist$, the less impact it has as compared to a standard RRT.  The larger $\bestneardist$ is, the harder it is to find narrow passages. We discuss further constraints on $\bestneardist$ in Section \ref{sec:select}.

To sample $\qrand_f = (\qrand_r,B^\textrm{rand})$, we sample the robot and band configurations independently, then combine the samples. For typical robot arms $\qrand_r$ is generated by sampling each joint independently and uniformly from the joint limits. To sample from $\bandspace$, we draw a sequence of $\maxbandpoints$ points from the bounded workspace. For our example tasks, workspace is a rectangular prism, and we sample each axis independently and uniformly.

Due to the fact that our system is highly underactuated, and the goal region is defined implicitly by a function call rather than an explicit set of configurations, we cannot sample from the goal set directly as is typically done for a goal bias. Instead we precompute a finite set of robot configurations $\Qapproxgoal$ such that the end-effectors of the robot are at $\eepositiongoal$. Then, as a goal bias mechanism, $\goalbias$ percent of the time, we attempt to connect to a potential goal configuration starting from the last configuration created by a call to the Connect function (or the last node selected by BestNearest if $\nodesnew = \emptyset$). A connection is then attempted between $\qlast_f$ and the nearest configuration in $\Qapproxgoal$. This allows us to bias exploration toward the robot component of the goal region, which we are able to define explicitly.




%%%%%%%%%%%%%%%%%%%%%%%%%%%%%%%%%%%%%%%%%%%%%%%%%%%%%%%%%%%%%%%%%%%%%%%%%%%%%%%%%%%%%%%%%%%%%%%%%%%%%%%%%%%%%%%%%%%%%%%

\section{Probabilistic Completeness of Global Planning}
\label{sec:analysis}

Proving probabilistic completeness in $\pspace$ is challenging due to the multi-modal nature of the problem. Specifically, as the virtual elastic band moves in and out of contact the dimensionality of the manifold that the system is operating in can change. In addition, the virtual elastic band forward propagation function (Alg.~\ref{alg:band_propogation}) can allow the band to ``snap tight'' as the grippers move past the edge of an obstacle, changing the number of points in the band representation as it does so. By leveraging the assumptions from Section~\ref{sec:rpath_assumptions}, we are able to bypass most of these challenges by focusing on the portion of $\pspace$ that can be analyzed; i.e. $\robotCspace$.

This section proves the probabilistic completeness of the planning approach in two major steps.  First, it will show that the approach for selecting the nearest node in the tree for expansion is equivalent to performing a nearest-neighbor query in the full space.  Second, it proves that our algorithm will eventually return a path that is $\deltar$-similar to an optimal $\delta$-robust solution to the planning problem with probability 1 (if it exists), or it will terminate early having found an alternate path to the goal region. Recall that we do not require an optimal path, only a feasible one.

\subsection{Assumptions and Definitions:}
\label{sec:rpath_assumptions}
Our problem allows for the virtual elastic band to be in contact with the surface of an obstacle, both during execution and as part of the goal set; this means that common assumptions regarding the expansiveness \cite{Hsu1999} of the planning problem may not hold. Instead of relying on expansiveness, we will define a series of alternate definitions and assumptions which are sufficient to ensure the completeness of our method.

First, in line with prior work, we will be assuming properties of the problem instance in regards to robustness.  In particular, we will be assuming the existence of a solution to a given query $\refpathf : [0, 1] \rightarrow \pspacevalid$ which has several robustness properties.  This solution is called a reference path.

To begin describing the properties of the reference path, we assume $\refpathf$ has robustness properties in the robot configuration space.  That is, the corresponding path in robot configuration space $\refpathr$ has strong $\deltar$-clearance under distance metric $d_\robotconfig(\cdot,\cdot)$ for some $\deltar > 0$.

\begin{definition}[Strong $\delta$-clearance]
    A path $\configpath : [0, 1] \rightarrow \pspacevalid$ has strong $\delta$-clearance under distance metric $d(\cdot,\cdot)$ if $\forall s \in [0, 1],\; d(\configpath(s), \pspaceinv) \geq \delta$, for $\delta > 0$.
\end{definition}

Given our assumption about the $\deltar$-clearance of the reference path in robot space, there exists a set $\setofpathsr$ of $\deltar$-similar paths to the reference path which are also collision-free.

\begin{definition}[$\delta$-similar path]
    Two paths $\configpath_a$ and $\configpath_b$ are $\delta$-similar if the Fr\'echet distance between the paths is less than or equal to $\delta$.
\end{definition}

Informally the Fr\'echet distance is described as follows~\cite{Alt1995Frechet}: Suppose a man is walking a dog. The man is walking on one curve while the dog on another curve. Both walk at any speed but are not allowed to move backwards. The Fr\'echet distance of the two curves is then the minimum length of leash necessary to connect the man and the dog.


Given the relationship between robot-space and full-space paths, we can define a full-space equivalent to $\setofpathsr$ as
\begin{equation}
    \setofpathsf = \{ \pspacepath \mid \rspacepath \in \setofpathsr \textrm{ and } \pspacepath = \textrm{FullSpace}(\rspacepath, \pstateinit) \} \enspace .
\end{equation}

Given these assumptions and definitions, we are ready to define an \textit{acceptable $\delta$-robust path}:
\begin{definition}[Acceptable $\delta$-Robust Path]
\label{def:robust}
A path $\refpathf$ is acceptable $\delta$-robust if the following hold:
\begin{enumerate}
    \item The robot-space reference path $\refpathr$ has strong $\deltar$-clearance for some $\deltar > 0$;
    \item The final state for every path $\pspacepath \in \setofpathsf$ is in $\Pspacegoal$. 
\end{enumerate}
\end{definition}
\noindent We assume there exists a reference path which satisfies this property and answers our given planning query:

\begin{assumption}[Solvable Problem]
    There exists some $\deltar > 0$ such that the planning problem admits an acceptable $\delta$-robust path.
    \label{ass:solvable_problem}
\end{assumption}

If a planning problem does not yield a reference path with this property, then it would be practically impossible for a sampling-based approach to solve it, as this would require sampling on a lower-dimensional manifold in robot space. Given that our planner is able to find paths, we believe this assumption is true except in special cases where the band must achieve a singular configuration to reach the goal.


While the focus of this paper is not on asymptotic optimality, we will make use of a cost function $\cost(\configpath)$ of a path in Section~\ref{sec:select}. Our cost function is path length in robot configuration space. With a cost function of this form we then assume from here onward that the reference path in question is optimal under the following definition.

\begin{definition}[Optimal $\delta$-Robust Path]
    Let $\setofpaths_{\pstate,\delta}$ be the set of all acceptable $\delta$-robust paths. A path $\refpathf$ is optimal $\delta$-robust if
    \begin{equation}
        \cost(\refpathf) = \inf_{\pspacepath \in \setofpaths_{\pstate,\delta}} \cost(\pspacepath) \enspace .
    \end{equation}
\end{definition}

Finally, we also assume that workspace is bounded. This will be true for any practical task and is rarely mentioned in the literature, but we will use this assumption in our analysis in Section~\ref{sec:nn_equiv}.

%%%%%%%%%%%%%%%%%%%%%%%%%%%%%%%%%%%%%%%%%%%%%%%%%%%%%%%%%%%%%%%%%
%% NN Equivalence
%%%%%%%%%%%%%%%%%%%%%%%%%%%%%%%%%%%%%%%%%%%%%%%%%%%%%%%%%%%%%%%%%
\subsection{Proof of Nearest-Neighbors Equivalence}
\label{sec:nn_equiv}

\begin{lemma}
    \label{lem:banddist}
     If the maximum distance between any two points in workspace is bounded by $\Dmaxw > 0$, then under distance metric $d_\band(\cdot, \cdot)$, the maximum distance between any two points in virtual elastic band space is bounded. I.e. $\exists \Dmaxb > 0$ such that $d_\band(\band_1, \band_2) \leq \Dmaxb \; \forall \band_1, \band_2 \in \bandspace$.
\end{lemma}

\noindent
{\bf Proof.}
From the definition of $\bandspace$ in Section~\ref{sec:overstretch}, the number of points used to represent a virtual elastic band is bounded by $\nbandpointsmax$. Let $\band_1, \band_2 \in \bandspace$ be two virtual elastic band configurations, and let $\tilde \band_1 = (\tilde b_{1,1}, \dots, b_{1,\nbandpointsmax})$ and $\tilde \band_2 = (\tilde b_{2,1}, \dots, b_{2,\nbandpointsmax})$ be their upsampled versions as described in Alg.~\ref{alg:band_dist}. Then
\begin{equation}
\begin{split}
    d_\band(\band_1, \band_2)^2 &= \sum_{i=1}^{\nbandpointsmax} \left\| \tilde b_{1,i} - \tilde b_{2,i} \right\|^2 \\
                                &\leq \sum_{i=1}^{\nbandpointsmax} \Dmaxw^2 = \nbandpointsmax \Dmaxw^2 = \Dmaxb^2
\end{split}
\end{equation}
\qed

\begin{lemma}
    If workspace is bounded, then lines \ref{alg:nearst:robotspace}-\ref{alg:nearest:fullspace} in Alg.~\ref{alg:nearest} are equivalent to a nearest neighbor search in the full configuration space directly.
\end{lemma}

\noindent
{\bf Proof.}
The upper bound of $\Dmaxb$ and our additive distance metric (Eq.~\eqref{eqn:dist_metric}) ensures that the distance between any two configurations in full space $\pspace$ can be bounded using only the distance in robot configuration space:
\begin{equation}
    d_\pstate(\cdot,\cdot)^2 \leq d_\robotconfig(\cdot,\cdot)^2 + \banddistscale \cdot \Dmaxb^2 \enspace .
    \label{eqn:bounded_config_distance}
\end{equation}
Next, consider that in Line~\ref{alg:nearst:robotspace} of the algorithm, the nearest neighbor to $\robotconfigrand$ under distance metric $d_\robotconfig$ is found.  Let this nearest neighbor be denoted $\nearapproxr$, keeping in mind that it belongs to a vertex in the tree $\nearapproxf = (\nearapproxr, \nearapproxb)$.  Let the (squared) distance between these points under $d_\robotconfig$ be $\Dnearr^2$.  From Eq.~\eqref{eqn:bounded_config_distance}, we can bound the distance between the random sample and $\nearapproxf$ under $d_\pstate$ as $\Dmaxf^2 \leq \Dnearr^2 + \banddistscale \Dmaxb^2 = \Dmaxf^2$.

In Line~\ref{alg:nearest:radius} of the algorithm, a radius nearest-neighbors query of radius $\Dmaxf$ is performed, returning a set $\Pstatenear$.  By construction if there is a node $\pstate \in \rrtnodeset$ that is closer to $\pstaterand$ than $\nearapproxf$, then $\pstate \in \Pstatenear$ (Figure~\ref{fig:nearest}). Then, the method selects as the true nearest neighbor in full space $\pstate^\textrm{select} = \argmin_{\pstate \in \Pstatenear}{d_\pstate(\pstate, \pstaterand)}$.
\qed

\begin{figure}[h]
    \centering
    \includegraphics[width=0.7\columnwidth]{nearest_neighbour}
    \caption{Left: $\robotconfig^{(2)}$ is the nearest node to the $\pstaterand$ in robot space, but it my be as far as $\Dmaxf$ away in the full configuration space. By considering all nodes within $\Dmaxf$ in robot space, we ensure that any node (such as $\pstate^{(1)}$) that is closer to $\pstaterand$ than $\pstate^{(2)}$ is selected as part of $\Pstatenear$, while nodes such as $\pstate^{(4)}$ are excluded in order to avoid the expense of calculating the full configuration space distance. Right: we then measure the distance in the full configuration space to all nodes that could possibly be the nearest to $\pstaterand$, returning $\pstate^{(1)}$ as the nearest node in the tree.}
    \label{fig:nearest}
\end{figure}

% \noindent
% {\bf Proof.} We will show this with a simple contradiction.  Assume there is a tree vertex $q^{nearest}_f$ not in $N$ which is the closest to $q^{rand}_f$, having minimal distance under metric $d_\pstate$ of ${D^{nearest}_f}^2$.  This implies ${D^{nearest}_f}^2 < {D^{near}_f}^2 (1)$.  Given the above bound, it must be that the distance satisfies ${D^{nearest}_f}^2 \leq {D^{nearest}_r}^2 + \banddistscale \cdot {D^{max}}^2 (2)$; however, since $q^{nearest}_f \notin N$, it must be the case that ${D^{nearest}_r}^2 > {D^{near}_r}^2 + \banddistscale \cdot {D^{max}}^2 (3)$.  By substituting $(1)$ into $(3)$, we get that $ {d_\robotconfig^{nearest}}^2 > {d_\pstate^{nearest}}^2 + \banddistscale \cdot {D^{max}}^2$; however, by substituting $(2)$ into this equation, we get ${d_\robotconfig^{nearest}}^2 > {d_\robotconfig^{nearest}}^2 + 2\banddistscale \cdot {D^{max}}^2$ implying $0 > 2\banddistscale \cdot {D^{max}}^2$, which is clearly false.  Since $(2)$ and $(3)$ are true by definition, $(1)$ must be false; therefore, any node not in $N$ cannot be the closest node to $q^{rand}_f$.  Since the method returns the node $q^{select}_f$ with minimal distance to $q^{rand}_f$ in $N$, $q^{select}_f$ must be the nearest neighbor of $q^{rand}_f$. 

%%%%%%%%%%%%%%%%%%%%%%%%%%%%%%%%%%%%%%%%%%%%%%%%%%%%%%%%%%%%%%%%%
%%  High-Level Proof
%%%%%%%%%%%%%%%%%%%%%%%%%%%%%%%%%%%%%%%%%%%%%%%%%%%%%%%%%%%%%%%%%

\subsection{Construction of a $\deltar$-similar Path}
\label{sec:delta_sim_traj_construction}

The objective here is to show with probability approaching $1$, the planner generates a $\deltar$-similar path to some robustly-feasible solution given enough time.  If an alternate path is found and the algorithm terminates before generating a $\deltar$-similar path then this is still sufficient for probabilistic completeness.  This analysis is similar to \cite{LiAOKP2016}, and is based on a covering ball sequence of the optimal  $\delta$-robust path $\refpathr$.

\begin{definition}[Covering Ball Sequence]
    \label{def:coveringballseq}
    Given a path $\rspacepath : [0, 1] \rightarrow \robotCvalid$, robust clearance $\deltar > 0$, a BestNearest distance $\bestneardist > 0$, and a distance value $0 < \ballseparation < \bestneardist < \deltar$; the covering ball sequence is defined as a set of $K + 1$ hyper-balls $\{ \hyperball_{\deltar}(\robotconfig_0), \dots, \hyperball_{\deltar}(\robotconfig_K) \}$ of radius $\deltar$, where $\robotconfig_k$ are defined such that:
    \begin{itemize}
        \item $\robotconfig_0 = \rspacepath(0)$;
        \item $\robotconfig_K = \rspacepath(1)$;
        \item $\textup{PathLength}(\robotconfig_{k-1}, \robotconfig_k) = \ballseparation$ for $k = 1, \dots, K$.
    \end{itemize}
\end{definition}
\noindent
Denote $\robotconfig^*_k$ to be the center of the $k^{th}$ covering hyper-ball for the reference path $\refpathr$. Figure~\ref{fig:covering_ball_sequence} shows an example of a covering ball sequence.

\begin{figure}[h]
    \centering
    \includegraphics[width=0.7\columnwidth]{CoveringBallSequence}
    \caption{Example covering ball sequence for an example reference path with a distance along the path of $\ballseparation$ between each ball. Given that the path is $\deltar$-robust, each ball is a subset of $\robotCvalid$.}
    \label{fig:covering_ball_sequence}
\end{figure}

The objective is to show that the vertex set of the planning tree after $n$ iterations $\rrtnodeset_n$ probabilistically contains a node within the goal set, i.e.
\begin{equation}
    \liminf_{n \rightarrow \infty} \pr( \rrtnodeset_n \cap \Pspacegoal \neq \emptyset ) = 1 \enspace .
\end{equation}
To do this, the analysis examines $K$ subsegments of the reference path $\refpathr$, based on the covering ball sequence for the reference path. If we can generate a robot path that is $\deltar$ similar to $\refpathr$, then given Assumption~\ref{ass:solvable_problem} and the properties of the reference path, the corresponding full space path will be a solution to the given planning problem.

Let $A_k^{(n)}$ be the event that on the $n^{th}$ iteration of the algorithm, it generates a $\deltar$-similar path to the $k^{th}$ subsegment of $\refpathr$.  This of course requires two events to occur: the node generated from the prior propagation covering segment $k-1$ must be selected for expansion, and the expansion must then produce a $\deltar$-similar path to the current segment.  Then, let $E_k^{(n)}$ be the event that for segment $k$, $A_k^{(n)}$ has occurred for some $i \in [1,n]$, i.e. $E_k^{(n)}$ indicates whether the algorithm has constructed the $\deltar$-similar edge for subsegment $k$. From these definitions, the goal then is to show that
\begin{equation}
    \lim_{n \rightarrow \infty} \pr(\textrm{Success}) = \lim_{n \rightarrow \infty} \pr\left( E^{(n)}_K \right) = 1 \enspace.
    \label{eqn:initial_pr_limit}
\end{equation}

We start by considering the probability of failing to generate an arbitrary segment $1 \leq k \leq K$. Then 
\begin{equation}
\begin{split}
    \pr&\left( \neg E^{(n)}_k \right) \\
       &= \pr\left( \neg A^{(1)}_k \cap \dots \cap \neg A^{(n)}_k \right) \\
       &= \pr\left( \neg A^{(1)}_k \right) \pr\left( \neg A^{(2)}_k \mid \neg A^{(1)}_k \right) \cdot \dots \\
       &\hspace{1.5cm} \cdot \pr\left( \neg A^{(n)}_k \mid \neg A^{(1)}_k \cap \dots \cap \neg A^{(n-1)}_k \right) \\
       &= \prod_{i=1}^n \pr\left( \neg A^{(i)}_k \mid \neg E^{(i-1)}_k \right) \enspace .
\end{split}
\label{eqn:fail_enk}
\end{equation}
Note the definition of $\neg E^{(i-1)}_k$ is what allows us to collapse the product into a concise form.


The probability that $\neg A^{(i)}_k$ happens given $\neg E^{(i-1)}_k$ is equivalent to the probability that we have not yet generated a $\deltar$-similar path for segment $k-1$ (i.e. $\pr( \neg E^{(i-1)}_{k-1} )$) plus the probability that the previous segment has been generated, but we fail to generate the current segment:
\begin{equation}
\begin{split}
    \pr&\left( \neg A^{(i)}_k \mid \neg E^{(i-1)}_k \right) \\
    &= \pr\left( \neg E^{(i-1)}_{k-1} \right) + \pr\left( E^{(i-1)}_{k-1} \right) \\
    &  \hspace{1.7cm} \cdot \pr\left( \neg A_k^{(i)} \mid E_{k-1}^{(i-1)} \cap \neg E_k^{(i-1)} \right), \\
\end{split}
\end{equation}
which we can rewrite in terms of $A_k^{(i)}$ instead of $\neg A_k^{(i)}$:
\begin{equation}
\begin{split}
    \pr&\left( \neg A^{(i)}_k \mid \neg E^{(i-1)}_k \right) \\
       &= \pr\left( \neg E^{(i-1)}_{k-1} \right) + \pr\left( E^{(i-1)}_{k-1} \right) \\
       &  \hspace{1.7cm} \cdot \left(1 - \pr\left( A_k^{(i)} \mid E_{k-1}^{(i-1)} \cap \neg E_k^{(i-1)} \right) \right) \enspace . \\
\end{split}
\end{equation}
Then multiplying out the last term we get
\begin{equation}
\begin{split}
    \pr&\left( \neg A^{(i)}_k \mid \neg E^{(i-1)}_k \right) \\
       &= \pr\left( \neg E^{(i-1)}_{k-1} \right) + \pr\left( E^{(i-1)}_{k-1} \right) \\
       &  \hspace{0.65cm} - \pr\left( E^{(i-1)}_{k-1} \right) \pr\left( A_k^{(i)} \mid E_{k-1}^{(i-1)} \cap \neg E_k^{(i-1)} \right) \enspace . \\
\end{split}
\end{equation}
Finally, summing the first two terms, we arrive at
\begin{equation}
\begin{split}
    \pr&\left( \neg A^{(i)}_k \mid \neg E^{(i-1)}_k \right) \\
       &= 1 - \pr\left( E^{(i-1)}_{k-1} \right) \pr\left( A_k^{(i)} \mid E_{k-1}^{(i-1)} \cap \neg E_k^{(i-1)} \right) \enspace .
\end{split}
\end{equation}
Two events need to happen in order to generate a path to the next hyperball; an appropriate node must be selected for expansion, and Connect$(\dots)$ must generate a $\deltar$-similar path segment, assuming that the appropriate node has already been selected. Denote the probability of these events at iteration $i$ as $\selectprobability$ and $\propagationprobability$ respectively. Then
\begin{equation}
    \pr\left( \neg A^{(i)}_k \mid \neg E^{(i-1)}_k \right) = 1 - \pr\left( E^{(i-1)}_{k-1} \right) \selectprobability \propagationprobability \enspace .
    \label{eqn:fail_aik}
\end{equation}

\noindent
As we are examining this probability in the limit, we will instead draw a bound on this probability to put it in a form we can easily examine the limit for. To do so, we must carefully consider the values of $\selectprobability$ and $\propagationprobability$.  In Section~\ref{sec:select}, it will be shown that $\selectprobability$ is a generally decreasing function, but converges to a finite value $\selectbound > 0$ in the limit.  Therefore we let $\selectbound$ be a lower bound of $\selectprobability$.  Then in Section~\ref{sec:prop}, $\propagationprobability$ will similarly be shown to be positive and lower-bounded; in particular $\selectprobability \propagationprobability \leq \selectbound$.  Taking $\selectbound$ as constant, we can bound Eq.~\eqref{eqn:fail_aik} as
\begin{equation}
    \pr\left( \neg A^{(i)}_k \mid \neg E^{(i-1)}_k \right) \leq 1 - \pr\left( E^{(i-1)}_{k-1} \right) \selectbound \enspace .
    \label{eqn:fail_aik_bound}
\end{equation}
Combining equations \eqref{eqn:fail_aik_bound} and \eqref{eqn:fail_enk} we have
\begin{equation}
    \pr\left( \neg E^{(n)}_k \right) \leq \prod_{i=1}^n \left(1 - \pr\left( E^{(i-1)}_{k-1} \right) \selectbound \right) \enspace .
\end{equation}
Denote $y^{(n)}_k = \prod_{i=1}^n \left(1 - \pr\left( E^{(i-1)}_{k-1} \right) \selectbound \right)$. Then
\begin{equation}
    \pr\left( \neg E^{(n)}_k \right) \leq y^{(n)}_k \enspace .
    \label{eqn:fail_enk_bound}
\end{equation}
We will show using induction over $k$, that Eq.~\eqref{eqn:fail_enk_bound} tends to 0 as $n \rightarrow \infty$, and thus $\lim_{n \rightarrow \infty} \pr(\textrm{Success}) = 1$\\

\noindent
\textbf{Base case $(k = 1)$:}\\
Note that $\pr(E^{(i)}_{0}) = 1$ because the start node always exists. Then 
\begin{equation}
\begin{split}
    \lim_{n \rightarrow \infty} \pr\left( \neg E^{(n)}_1 \right) 
        &\leq \lim_{n \rightarrow \infty} \prod_{i=1}^n \left(1 - \pr\left( E^{(i-1)}_{0} \right) \selectbound \right) \\
        &=    \lim_{n \rightarrow \infty} \prod_{i=1}^n \left(1 - \selectbound \right) \\
        &=\lim_{n \rightarrow \infty} \left(1 - \selectbound \right)^n = 0 \enspace .
\end{split}
\end{equation}

\noindent
\textbf{Induction hypothesis: }
\begin{equation}
    \lim_{n \rightarrow \infty} \pr\left( \neg E^{(n)}_m \right) = 0 \textrm{ for } m = 1, 2, \dots, k - 1 \enspace .
    \label{eqn:induction_hypothesis}
\end{equation}
Note that this implies $\lim_{n \rightarrow \infty} \pr( E^{(n)}_m ) = 1$ for $m = 1, 2, \dots, k - 1$.\\

\noindent
\textbf{Induction step ($2 \leq k \leq K$): }\\
Consider the log of the bound on $\pr\left( \neg E^{(n)}_k \right)$:
\begin{equation}
    \log y^{(n)}_k = \sum_{i=1}^n \log\left( 1 - \pr\left( E^{(i-1)}_{k-1} \right) \selectbound \right) \enspace .
    \label{eqn:logy}
\end{equation}
Denote $x = \pr\left( E^{(i-1)}_{k-1} \right) \selectbound$. Given that $0 \leq x < 1$, and writing the Taylor series expansion of $\log\left( 1 - x \right)$ centered at $x = 0$ we have
\begin{equation}
    \log \left( 1 - x \right) = - \sum_{m=1}^\infty \frac{x^m}{m} \enspace .
    \label{eqn:log1minusx}
\end{equation}
Substituting Eq.~\eqref{eqn:log1minusx} back into Eq.~\eqref{eqn:logy} we get
\begin{equation}
    \log y^{(n)}_k = - \sum_{i=1}^n \sum_{m=1}^\infty \frac{\left( \pr\left( E^{(i-1)}_{k-1} \right) \selectbound \right)^m}{m} \enspace .
\end{equation}
Dropping all but the first term in the infinite sum we get the bound
\begin{equation}
    \log y^{(n)}_k \leq -\sum_{i=1}^n \pr\left( E^{(i-1)}_{k-1} \right) \selectbound \enspace .
\end{equation}
Rearranging terms yields
\begin{equation}
    \log y^{(n)}_k \leq -\selectbound \sum_{i=1}^n \pr\left( E^{(i-1)}_{k-1} \right) \enspace .
\end{equation}
We now use the induction hypothesis. We know that $\pr( E^{(n)}_{k-1} ) \rightarrow 1$ as $n \rightarrow \infty$, thus $\sum_{i=1}^n \pr( E^{(i-1)}_{k-1} ) \rightarrow \infty$. Then
\begin{equation}
    \lim_{n \rightarrow \infty} \log y^{(n)}_k \leq -\selectbound \sum_{i=1}^n \lim_{n \rightarrow \infty} \pr\left( E^{(i-1)}_{k-1} \right) = -\infty \enspace .
    \label{eqn:logy_bound}
\end{equation}
Taking the log of Eq.~\eqref{eqn:fail_enk_bound} and combining with Eq.~\eqref{eqn:logy_bound} we get
\begin{equation}
    \lim_{n \rightarrow \infty} \log \pr\left( \neg E^{(n)}_k \right) \leq \lim_{n \rightarrow \infty} \log y^{(n)}_k = -\infty
\end{equation}
and therefore
\begin{equation}
    \lim_{n \rightarrow \infty} \pr\left( \neg E^{(n)}_k \right) = 0 ,
\end{equation}
which completes the induction step.


Thus, given that $\pr(\neg E^{(n)}_k) \rightarrow 0$ as $n \rightarrow \infty$ for any $1 \leq k \leq K$
\begin{equation}
    \lim_{n \rightarrow \infty} \pr(\textrm{Success}) = \lim_{n \rightarrow \infty} \left( 1 - \pr\left( \neg E^{(n)}_K \right) \right) = 1 \enspace.
\end{equation}



%%%%%%%%%%%%%%%%%%%%%%%%%%%%%%%%%%%%%%%%%%%%%%%%%%%%%%%%%%%%%%%%%
%% Node Selection
\subsubsection{Selection of an appropriate node ($\selectbound$):}
\label{sec:select}

First, we define the following restriction on the definition of $\bestneardist$: 

\begin{definition}[$\bestneardist$ Restriction]
\label{prop:bestnear_requirement}
    For a reference path $\refpathr$ with robustness $\deltar$, $\bestneardist$ is defined such that $\innerballsize = \deltar - \bestneardist > 0$.
\end{definition}

The proof that $\selectbound > 0$ follows directly from the related work of \cite{LiAOKP2016} (proof of Lemma 23).  To summarize, due to best-nearest neighbors selection, there exists a positive-measure region around the minimum cost vertex $\pstatenear$ which observes the optimal reference path in which its cost dominates all other nearby nodes, and therefore, when $\pstaterand$ is drawn in this volume, $\pstatenear = (\pstatenearfull)$ is guaranteed to be selected (Figure~\ref{fig:Yanbo_lemma_23_figure}).  Since our approach follows an equivalent sampling and nearest neighbor method to \cite{LiAOKP2016} (as shown in Section~\ref{sec:nn_equiv}), 
\begin{equation}
    \selectbound = \frac{\mu\left( \hyperball_{\innerballsize}\big( \pstate_k^* \big) \cap \hyperball_{\bestneardist}\big( \pstatenear \big) \right)}{\mu\left( \pspace \right)} > 0
\end{equation}
follows directly.

\begin{figure}[h]
    \centering
    \includegraphics[width=0.3\columnwidth]{nearest_neighbour_domination_region}
    \caption{Minimum domination region for a node $\pstate_i$, adapted from \citet{LiAOKP2016} Lemma 23. Sampling $\pstaterand$ in the shaded region guarantees that a node $\pstatenear \in \hyperball_{\deltar}(\pstate_k^*)$ is selected for propagation so that either $\pstatenear = \pstate_i$ or $\cost(\pstatenear) < \cost(\pstate_i)$.}
    \label{fig:Yanbo_lemma_23_figure}
\end{figure}

To show that $\selectbound < 1$, we need only consider the case when there are at least 2 nodes in $\rrtnodeset$.

%%%%%%%%%%%%%%%%%%%%%%%%%%%%%%%%%%%%%%%%%%%%%%%%%%%%%%%%%%%%%%%%%
%% Node Propagation
%%%%%%%%%%%%%%%%%%%%%%%%%%%%%%%%%%%%%%%%%%%%%%%%%%%%%%%%%%%%%%%%%
\subsubsection{$\deltar$-similar Propagation ($\propagationprobability$):}
\label{sec:prop}
Given that our nearest neighbor method is non-standard, and operating in the full configuration space $\pspace$, we need to carefully consider how this affects the propagation probability $\propagationprobability$. Given the kinematic model of our robot system, it is straightforward to show that the system in robot space is Small-Time Locally Controllable (STLC), i.e. $\robotconfig$ can be instantaneously moved in any direction, barring the presence of obstacles or configuration space limits.  

Then, based on the construction of the covering ball sequence and the $\bestneardist$ restriction, the following lemma holds.

\begin{lemma}
    \label{lem:rand_to_next_dist}
    If $\pstaterand$ is within the minimum domination region as described in \cite{LiAOKP2016} Lemma 23 (Figure~\ref{fig:Yanbo_lemma_23_figure}), then $\robotconfigrand \in \hyperball_{\deltar}(\robotconfig^*_k)$ and Connect() will generate a segment that is $\deltar$-similar to segment $k$ of the reference path.
\end{lemma}



\noindent
{\bf Proof.}
Assume that $\pstaterand \in \hyperball_{\innerballsize}(\pstate^*_{k-1})$. Then we have
\begin{align*}
    d_\robotconfig(\robotconfigrand, \robotconfig^*_k)  &\leq d_\pstate(\pstaterand, \pstate^*_k) \\
                                                        &\leq d_\pstate(\pstaterand, \pstate^*_{k-1}) + d_\pstate(\pstate^*_{k-1}, \pstate^*_k) \\
                                                        &\leq \innerballsize + \ballseparation = \deltar - \bestneardist + \ballseparation \enspace.
\end{align*}
Then by construction of the covering ball sequence, we have that $\ballseparation -\bestneardist < 0$ and thus $d_\robotconfig(\robotconfigrand, \robotconfig^*_k) < \deltar$. In addition, we have that the straight line between $\robotconfignear$ as selected by $\robotconfigrand$ is entirely contained in $\hyperball_{\deltar}(\robotconfig^*_{k-1})$, and thus is also in $\robotCvalid$ as the reference path is optimal $\delta$-robust. We then have that the path generated by Connect is $\deltar$-similar to the $k^{th}$ segment of the reference path.
\qed

\begin{lemma}
    The probability of covering segment $k$ at iteration $i$, given that we have not yet covered segment $k$ but we have covered segment $k-1$
    $$\pr\left( A_k^{(i)} \mid E_{k-1}^{(i-1)} \cap \neg E_k^{(i-1)} \right) = \selectprobability \propagationprobability$$
    is lower-bounded by $\selectbound$.
\end{lemma}

\noindent
{\bf Proof.}
Consider two possible events. First, that $\pstaterand$ is within the minimum domination region (Figure~\ref{fig:Yanbo_lemma_23_figure}) of $\Pstatenear$. If $\pstaterand$ is within the minimum domination region of $\Pstatenear$, then by Lemma~\ref{lem:rand_to_next_dist}, Connect() will generate a $\deltar$-similar segment with probability 1. Denote this event as $B$. Second, the event that $\pstaterand$ is somewhere else. Denote this event as $C$. Then we can bound $\pr( A_k^{(i)} \mid E_{k-1}^{(i-1)} \cap \neg E_k^{(i-1)})$ by considering only $B$:
\begin{align*}
    \pr\left( A_k^{(i)} \mid E_{k-1}^{(i-1)} \cap \neg E_k^{(i-1)} \right) 
        &= \pr(B) + \pr(C) \\
        &\geq \pr(B) \geq \selectbound \enspace .
\end{align*}
\qed

\input{5_experiments}
