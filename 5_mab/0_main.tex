\chapter{Bandit-Based Model Selection}

In the previous chapters, we have been working with a single model and a single controller for any given task. When given a new task however, a new choice needs to be made for what model and controller is most suitable. Rather than assuming we have a single high-fidelity model of a deformable object interacting with its environment, our approach is to have multiple models available for use, any one of which may be useful at a given time. We do not assume these models are correct, we simply treat the models as having some measurable \textit{utility} to the task. The \textit{utility} of a given model is the expected reduction in task error when using this model to generate robot motion. As the task proceeds, the utility of a given model may change, making other models more suitable for the current part of the task. However, without testing a model's prediction, we do not know its true utility. Testing every model in the set is impractical, as all models would need to be tested at every step, and performing a test changes the state of the object and may drive it into a local minimum. The key question is then which model should be selected for testing at a given time.

The central contribution of this chapter is framing the model selection problem as a \ac{MAB} problem where the goal is to find the model that has the highest utility for a given task. An arm represents a single model of the deformable object; to ``pull'' an arm is to use the arm's model to generate and execute a velocity command for the robot. The reward received is the reduction in task error after executing the command. In order to determine which model has the highest utility we need to explore the model space, however we also want to exploit the information we have gained by using models that we estimate to have high utility. One of the primary challenges in performing this exploration versus exploitation trade-off is that our models are inherently coupled and non-stationary; performing an action changes the state of the system which can change the utility of every model, as well as the reward of pulling each arm. While there is work that frames robust trajectory selection as a \ac{MAB} problem~\cite{Koval2015}, we are not aware of any previous work which either 1) frames model selection for deformable objects as a \ac{MAB} problem; or 2) addresses the coupling between arms for non-stationary \ac{MAB} problems.

In our experiments, we show how to formulate a \ac{MAB} problem with coupled arms for Jacobian-based models. We perform our experiments on three synthetic systems, and on three deformable object manipulation tasks in the Bullet Physics~\cite{Coumans2010} simulator. We demonstrate that formulating model selection as a \ac{MAB} problem is able to successfully perform all three manipulation tasks. We also show that our proposed \ac{MAB} algorithm outperforms previous \ac{MAB} methods on synthetic trials, and performs competitively on the manipulation tasks.

\section{Problem Statement}
\label{sec:main_problem_statement}

\todoin{This duplicates some previous stuff. Refer back or keep here?}
Define the robot configuration space to be $\robotCspace$. We assume that the robot configuration can be measured exactly. Denote an individual robot configuration as $\robotconfig \in \robotCspace$. This set can be partitioned into a valid and invalid set. The valid set is referred to as $\robotCvalid$, and is the set of configurations where the robot is not in collision with the static geometry of the world. The invalid set is referred to as $\robotCinvalid = \robotCspace \setminus \robotCvalid$.

\todoin{Confirm the `previous chapters' statement.}
We assume that our model of the robot is purely kinematic, with no higher order dynamics. Previous chapters assumed an arbitrary number of grippers; in this chapter we restrict the problem to cases where two end-effectors that are rigidly attached to the object. 
% The configuration of a deformable object is a set $\deformconfig \subset \reals^3$ of $\numdeformpoints = | \deformconfig |$ points. We assume that we have a method of sensing $\deformconfig$. The rest of the environment is denoted $\obstacle$ and is assumed to be both static, and known exactly. 
We assume that the robot moves slowly enough that we can treat the combined robot and deformable object as quasi-static. Let the function $\DeformForwardFn(\robotconfig, \deformconfig, \robotvel)$ map the system configuration $(\robotconfig, \deformconfig)$ and robot movement $\robotvel$ to the corresponding deformable object movement $\deformvel$.

Similar to the previous chapter, we define a task based on a set of $\ntargetpoints$ target points $\target \subset \reals^3$, a function $\ErrorFnFull \rightarrow \reals^{\geq 0}$, which measures the alignment error between $\deformconfig$ and $\target$, and a termination function $\TermCondFull$ which indicates if the task is finished. Let a robot controller be a function $\CtrlFull$\footnote{A specific controller may have additional parameters (such as gains in a PID controller), but we do not include such parameters here to keep $\Ctrl(\dots)$ in a more general form.} which maps the system state $\left( \robotconfig, \deformconfig \right)$ and alignment targets $\target$ to a desired robot motion $\robotvelcmd$. In this work we restrict our discussion to tasks and controllers of the form introduced in Chapter~\ref{chap:local_control}; these controllers are local, i.e. at each time $t$ they choose an incremental movement $\robotvelcmd$ which reduces the alignment error as much as possible at time $t + 1$. 

The problem we address in this chapter is how to find a sequence of $\taskexecutiontime$ robot commands $\robotvelcmdexpanded = \robotvelcmdseq$ such that each motion is feasible, i.e. it should not bring the grippers into collision with obstacles, should not cause the object to stretch excessively, and should not exceed the robot's maximum velocity $\robotvelmax$. Let these feasibility constraints be represented by $A(\robotvelcmdseq) = 0$. Then the problem we seek to solve is:
% \begin{equation}
%     \begin{aligned}
%         & \text{find}   & & \robotcommandsequence \\
%         & \text{s.t.}   & & \terminationcondition(\deformconfig_{\taskexecutiontime}) = \texttt{true} \\
%         &               & & A(\robotcommandvel[t])                                    = 0, \; t = 1, \dots, \taskexecutiontime
%     \end{aligned}
%     \label{eqn:main_problem_statement}
% \end{equation}
\begin{equation}
    \begin{aligned}
        & \text{find}   & & \taskexecutiontime, \robotvelcmdseq \\
        & \text{s.t.}   & & \TermCond(\target, \deformconfig_{\taskexecutiontime}) = \texttt{true}\\ 
        &               & & A(\robotvelcmdseq) = 0\\
    \end{aligned}
    \label{eqn:veb_general_problem}
\end{equation}
where $\deformconfig_{\taskexecutiontime}$ is the configuration of the deformable object after executing $\robotvelcmdseq$.

Solving this problem directly is impractical in the general case for two major reasons. First, modeling a deformable object accurately is very difficult in the general case, especially if it contacts other objects or itself. Second, even given a perfect model, computing precise motion of the deformable object requires physical simulation, which can be very time consuming inside a planner/controller where many potential movements need to be evaluated. We seek a method which does not rely on high-fidelity modelling and simulation; instead we present a framework combining both global planning and local control to leverage the strengths of each in order to efficiently perform the task.

\input{2_bandit_based_model_selection}
\section{Multi-Armed Bandit Formulation for Deformable Object Manipulation}

\todoin{Rename this section or the whole chapter}
\todoin{Update main loop to use correct function names per previous chapter}

\begin{algorithm}[ht]
    \caption{MainLoop$(\obstacle, \beta, \lambda)$}
    \begin{algorithmic}[1]
        \State $t \gets 0$
        \State $\RelaxedDistMatrix \gets$ GeodesicDistanceMatrix$(\deformconfig_{relaxed})$
        \State $\modelset \gets$ InitializeModels$(\RelaxedDistMatrix)$
        \State InitialzeBanditAlgorithm()
        \State $\deformconfig(0) \gets$ SensePoints()
        \State $\robotconfig(0) \gets$ SenseRobotConfig()
        \While{true}
            \State $\modelchosen \gets $ SelectArmUsingBanditAlgorithm()
            
            \State $\target \gets$ GetTargets()
            \State $\correspondences \gets$ CalculateCorrespondences$(\deformconfig_t, \target)$
            \State $\deformvel_e, \Pinvweight_e \gets$ FollowNavigationFunction$(\deformconfig_n, \correspondences)$
            \State $\deformvel_s, \Pinvweight_s \gets$ StretchingCorrection$(\RelaxedDistMatrix, \stretchmax, \deformconfig)$
            \State $\deformvel_d, \Pinvweight_d \gets$ CombineTerms$(\deformvel_e, \Pinvweight_e, \deformvel_s, \Pinvweight_s, \stretchweightfactor)$

            \State $\grippervel_d \gets \DeformBackwardFn_m(\deformvel_d, \Pinvweight_d)$
            \State $\grippervel \gets$ ObstacleRepulsion$(\grippervel_d, \obstacle, \obsavoidfactor)$
            
            \State CommandConfiguration$(\gripperconfig(t) + \grippervel)$

            \State $\deformconfig(t + 1) \gets$ SensePoints$()$
            \State $\robotconfig(t + 1) \gets$ SenseRobotConfig$()$
            \State UpdateBanditAlgorithm$()$
            
            \State $t \gets t + 1$
        \EndWhile
    \end{algorithmic}
    \label{alg:mab_mainloop}
\end{algorithm}

\todoin{Move Fig.~\ref{fig:distance} to modelling section.}
\begin{figure}[ht]
    \centering
    \includegraphics[height=1.5in]{EuclideanVsGeodesic}
    \caption{Euclidean distance measures length of the shortest path between $p_i$ and $p_j$ in $\reals^3$ (gold). Geodesic distance measures the length of the shortest path, constrained to stay within the deformable object (red).}
    \label{fig:distance}
\end{figure}

Our algorithm~(Alg.~\ref{alg:mab_mainloop}) can be broken down into four major sections and an initialization block. In the initialization block we pre-compute the geodesic distance (see Fig.~\ref{fig:distance}) between every pair of points in $\deformconfig$ when the deformable object is in its ``natural'' or ``relaxed'' state and store the result in $\RelaxedDistMatrix$. These distances are used to construct the deformation models~(Sec.~\ref{sec:jacobian_models}), as well as to avoid overstretching the object (Sec.~\ref{sec:stretching_avoidance_old}).
At each iteration we: 
\begin{enumerate}
    \item pick a model to use to achieve the desired direction~(Sec.~\ref{sec:bandit_algorithms}); 
    \item compute the task-defined desired direction to move the deformable object~(Sec.~\ref{sec:reducing_error}); 
    \item generate a velocity command using the chosen model~(Sec.~\ref{sec:stretching_avoidance_controller}); 
    \item modify the command to avoid obstacles~(Sec.~\ref{sec:stretching_avoidance_controller});
    \item update bandit algorithm parameters~(Sec.~\ref{sec:bandit_algorithms}).
\end{enumerate}



\input{4_mab_algorithms}
\input{5_results}
